\documentclass{beamer}
\usetheme{Madrid} % Вы можете выбрать другую тему по желанию
\usepackage[utf8]{inputenc}
\usepackage[russian]{babel}

\title{Многопроцессорное расписание}
\author{Габдрахманов Азат}
\date{\today}

\begin{document}

% Титульный слайд
\begin{frame}
    \titlepage
\end{frame}

% Введение
\begin{frame}{Введение}
    \begin{itemize}
        \item Многопроцессорное расписание — важная задача в области теории расписаний и оптимизации.
        \item Применяется в компьютерных системах, производственных процессах, логистике и других областях.
        \item Цель: оптимальное распределение заданий между процессорами для достижения наилучшего результата.
        % \item https://en.wikipedia.org/wiki/Flow-shop_scheduling
    \end{itemize}
\end{frame}

% Формальное определение задачи
\begin{frame}{Формальное определение задачи}
    \textbf{Условие:}
    \begin{itemize}
        \item Задано множество заданий \( T \)
        \item Число процессоров \( m \)
        \item Длительность каждого задания \( l(t) \) для \( t \in T \)
        \item Общий директивный срок \( D \)
    \end{itemize}
    
    \textbf{Вопрос:}
    Существует ли \( m \)-процессорное расписание для заданий из \( T \), которое удовлетворяет общему директивному сроку \( D \)?
    
    \textbf{Формальное условие:}
    \[
    \exists \, o: T \rightarrow \mathbb{Z}, \quad \forall u \geq 0, \quad \#\{ t \in T \mid o(t) \leq u < o(t) + l(t) \} \leq m
    \]
    и
    \[
    \forall t \in T, \quad o(t) + l(t) \leq D
    \]
\end{frame}

% Общие виды заданий
\begin{frame}{Общие виды заданий}
    \begin{itemize}
        \item \textbf{Процессоры с разной производительностью:}
        \begin{itemize}
            \item Некоторые процессоры могут выполнять задания быстрее других.
            \item Время выполнения задания на процессоре \( P_j \): \( l(t)/s_j \), где \( s_j \) — скорость процессора.
        \end{itemize}
        \item \textbf{Зависимости между заданиями:}
        \begin{itemize}
            \item Некоторые задания не могут начинаться до завершения других.
        \end{itemize}
        \item \textbf{Приоритеты заданий:}
        \begin{itemize}
            \item Задания имеют разные приоритеты, влияющие на порядок их выполнения.
        \end{itemize}
    \end{itemize}
\end{frame}

% Частные случаи задачи
\begin{frame}{Частные случаи задачи}
    \begin{itemize}
        \item \textbf{Одинаковое время выполнения всех заданий:}
        \begin{itemize}
            \item \( l(t) = l \) для всех \( t \in T \)
            \item Проблема становится аналогом задачи разбиения множества (Partition)
            \item В этом случае задача может быть решена за полиномиальное время.
        \end{itemize}
        \item \textbf{Ограниченное количество процессоров:}
        \begin{itemize}
            \item Для фиксированного \( m \) задача может быть проще решаемой.
        \end{itemize}
        \item \textbf{Отсутствие дедлайнов:}
        \begin{itemize}
            \item Задача упрощается, если нет ограничений по времени завершения.
        \end{itemize}
    \end{itemize}
\end{frame}

% Доказательство NP-полноты: Принадлежность к классу NP
\begin{frame}{Доказательство NP-полноты: Принадлежность к классу NP}
    \textbf{Класс NP:}
    \begin{itemize}
        \item Включает задачи, решения которых можно \textbf{проверить} за полиномиальное время.
    \end{itemize}

    \textbf{Сертификат для задачи многопроцессорного расписания:}
    \begin{itemize}
        \item Функция назначения \( o: T \rightarrow \mathbb{Z} \), определяющая время начала выполнения каждого задания.
    \end{itemize}

    \textbf{Процесс проверки корректности расписания:}
    \begin{enumerate}
        \item \textbf{Проверка выполнения всех заданий до директивного срока \( D \):}
        \[
        \forall t \in T, \quad o(t) + l(t) \leq D
        \]
        \item \textbf{Проверка ограничения на количество одновременно выполняющихся заданий:}
        \[
        \forall u \geq 0, \quad \#\{ t \in T \mid o(t) \leq u < o(t) + l(t) \} \leq m
        \]
        \begin{itemize}
            \item Используем алгоритм сортировки событий (начало и конец выполнения заданий).
            \item Проверяем, что в любой момент времени выполняется не более \( m \) заданий.
        \end{itemize}
    \end{enumerate}

    \textbf{Заключение:} Поскольку оба шага выполняются за полиномиальное время, задача принадлежит классу \textbf{NP}.
\end{frame}

% Доказательство NP-полноты: NP-Трудность
\begin{frame}{Доказательство NP-полноты: NP-Трудность}
    \textbf{Класс NP-трудных задач:}
    \begin{itemize}
        \item Задачи, к которым \textbf{любые} задачи из NP могут быть сведены за полиномиальное время.
    \end{itemize}

    \textbf{Сведение из задачи Partition:}
    \begin{itemize}
        \item \textbf{Задача Partition:}
        \begin{itemize}
            \item Дано множество чисел \( S = \{s_1, s_2, \dots, s_{2m}\} \) с суммой \( 2K \).
            \item Вопрос: Можно ли разбить \( S \) на два подмножества \( S_1 \) и \( S_2 \) такие, что \( \sum_{s \in S_1} s = \sum_{s \in S_2} s = K \)?
        \end{itemize}
        \item Partition является NP-полной задачей.
    \end{itemize}

    \textbf{Сведение Partition к многопроцессорному расписанию:}
    \begin{enumerate}
        \item \textbf{Построение экземпляра задачи расписания:}
        \begin{itemize}
            \item \textbf{Задания \( T \):} Для каждого числа \( s_i \in S \) создаём задание \( t_i \) с длительностью \( l(t_i) = s_i \).
            \item \textbf{Количество процессоров \( m' = 2 \).}
            \item \textbf{Директивный срок \( D = K \).}
        \end{itemize}
        \item \textbf{Эквивалентность решений:}
        \begin{itemize}
            \item Если существует расписание, завершающее все задания до \( K \) на 2 процессорах, то это соответствует разбиению \( S \) на \( S_1 \) и \( S_2 \) с суммой \( K \) в каждом.
            \item Обратное также верно.
        \end{itemize}
    \end{enumerate}

    \textbf{Заключение:} Поскольку задача Partition сводится к задаче многопроцессорного расписания за полиномиальное время, задача многопроцессорного расписания является \textbf{NP-трудной}.
\end{frame}

% Применение задачи
\begin{frame}{Применение задачи многопроцессорного расписания}
    \begin{itemize}
        \item \textbf{Компьютерные системы:}
        \begin{itemize}
            \item Распределение процессов и потоков между ядрами процессора.
        \end{itemize}
        \item \textbf{Производственные системы:}
        \begin{itemize}
            \item Планирование производственных процессов на различных машинах или линиях.
        \end{itemize}
        \item \textbf{Логистика:}
        \begin{itemize}
            \item Организация доставки и распределения ресурсов.
        \end{itemize}
        \item \textbf{Облачные вычисления:}
        \begin{itemize}
            \item Оптимизация использования виртуальных машин и серверов.
        \end{itemize}
    \end{itemize}
\end{frame}

% Заключение
\begin{frame}{Заключение}
    \begin{itemize}
        \item Многопроцессорное расписание — фундаментальная задача в оптимизации и теории сложности.
        \item Задача NP-полная, что делает её решение для больших экземпляров вычислительно сложным.
        \item Практическое применение в различных областях требует использования эвристических и приближенных методов.
        \item Понимание сложности задачи важно для разработки эффективных алгоритмов распределения ресурсов.
    \end{itemize}
\end{frame}

% Список литературы
\begin{frame}{Список литературы}
    \begin{thebibliography}{9}
        \bibitem{garey1979computers}
        M. R. Garey, D. S. Johnson. \textit{Computers and Intractability: A Guide to the Theory of NP-Completeness}. W.H. Freeman, 1979.
        
        \bibitem{scheduling}
        R. Graham. \textit{Scheduling: Theory, Algorithms, and Systems}. Prentice Hall, 1966.
        
        \bibitem{multiprocessor}
        K. Pruhs, J. R. Rice. \textit{Scheduling on Multiprocessors: Algorithms and Complexity}. Springer, 1997.
    \end{thebibliography}
\end{frame}

\end{document}
