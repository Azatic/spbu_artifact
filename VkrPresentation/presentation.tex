\documentclass{vkr-slides-style}

\filltitle{
    % Ваше ФИО.
    author = {Азат Райнурович Габдрахманов},
    % Нельзя оставлять пустые строки.
    % Ваше сокращённое именование, будет показываться слева внизу слайда.
    authorShort = {Азат Габдрахманов},
    %
    % Официальное название ВКР.
    title = {Разработка веб приложения для шахматной школы},
    %
    % Короткое название ВКР, будет снизу по центру слайда.
    titleShort = {Шахматный сайт},
    %
    % Научный руководитель.
    advisor = {к.ф.-м.н. Д.В. Луцив, доцент кафедры системного программирования},
    %
    % Консультант (если есть). Если нет, оставьте пустые фигурные скобки.
    consultant = {},
    %
    % Дата доклада.
    date = {23.11.2024}
}
 
\begin{document}

\makeslidestitle

% \begin{frame}  
%     \frametitle{Суть работы}
%     \begin{itemize}
%         \item Тут кратко и по делу рассказываете, что именно хотите сделать
%         \item Обозначьте отчуждаемый результат (что это будет --- приложение, инструмент, библиотека, ...), открыт ли код
%         \item Должно быть понятно, в чём решаемая проблема, почему нельзя взять готовое решение,
%          и кому это нужно (желательно с точностью до конкретной компании, а не вообще --- если работодатель предложил или утвердил тему, укажите это тут)
%         \item Должен быть понятен объём работы (почему это диплом)
%         \item Должно быть понятно, при чём тут СП\footnote{Ещё не поздно защищаться на ИАС или информатике.}
%     \end{itemize}
% \end{frame}


\begin{frame}  
    \frametitle{Суть работы}
    \begin{itemize}
        % \item Идеей является создание приложения, которое будет включать себя всю нужную для тренера инфраструктуру для преподавания.

        \item Максимально облегчить рутину тренера, предоставив удобный инструмент для создания домашних заданий по любым темам, легкой проверки результатов учеников и анализа их прогресса.
        
        \item Анализ шахматных партий в комнатах с движком или без\footnote{Это важно!}
        
        \item Хранение шахматной базы знаний школы.
        
        \item Работа включает сбор функциональных требований, обзор аналогов, реализация и аппробация.
        
        % \item Проект связан с веб-разработкой, с использованием современных технологий (React, ASP.NET Core, Postgresql) для конкретной шахматной школы.
    \end{itemize}
\end{frame}


\begin{frame}  
    \frametitle{Постановка задачи}
    \textbf{Цель:} Создание веб-приложения для шахматной школы

    \vspace{5mm}
    \textbf{Задачи:}
    \begin{enumerate}
        \item Сформулироавть требования
        \item Выполнить обзор аналогов
        \item Спроектировать решение
        \item Реализовать веб-приложение
        \item Выполнить апробацию
    \end{enumerate}
\end{frame}

\begin{frame}  
    \frametitle{План работы}
    \textbf{Что уже сделано:}
    \begin{enumerate}
        \item Выполнен обзор аналогов
        \item Собраны требования
        \item Реализована часть функциональности MVP
    \end{enumerate}

    \textbf{Планируется к зимней защите:}
    \begin{enumerate}
        \setcounter{enumi}{1}
        \item Доделать MVP
        \item Провести апробацию MVP и приступить к реализации дальнейших требований заказчика
    \end{enumerate}

    \textbf{Планируется к защите ВКР:}
    \begin{enumerate}
        \setcounter{enumi}{3}
        \item Реализовать все требования продукта
        \item Выполнить апробацию
    \end{enumerate}
\end{frame}

\end{document}